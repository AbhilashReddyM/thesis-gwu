% !TEX root = ../thesis-sample.tex

% --------- FRONT MATTER PAGES ---------------------
% Title of the thesis
\title{I guess it's better to be lucky than good.}

% Author name
\author{William T. Riker}

% Previous degrees
\bsdepartment{Clean Shaving}
\bsschool{The School of Hard Knocks}
\bsgrad{May 2350}

\msdepartment{Epic Beards}
\msschool{Starfleet Academy}
\msgrad{May 2353}
\showmsdegree % you can show or hide the MS degree line 
% \hidemsdegree

% PhD degree commands
% Committee
\showcommitteepage % hide this page if you're doing a MS thesis
%\hidecommitteepage 
\committee{ %
Leonhard Euler, Associate Professor of Engineering and Applied Science,\\ 
Dissertation Director\\ % remember to add a space between committee members

Full Name, Title, \\
Dissertation Director/Dissertation Co-Director/Committee Member
}

% Chair must be entered separately for formatting reasons.
\chair{Leonhard Euler}
\chairtitle{Associate Professor of Mechanical and Aerospace Engineering}
% Department
\department{Mechanical and Aerospace Engineering}

\phdgrad{May 28, 2358}
\defensedate{May 28, 2358}
% Year of completion for copyright page and perhaps other places
\year=2358 

% Copyright page
%\copyrightholder{Someone else}

% Dedication
\dedication[8]{ %
\textit{Fear is the true enemy, the only enemy.}
}

% Acknowledgments
\acknowledgments{
These are the voyages of the Starship Enterprise. Its continuing mission, to explore strange new worlds, to seek out new life and new civilizations, to boldly go where no one has gone before.
}

% -----------------------------------------------------------------
% Typically only one of Preface/Foreward/Prologue would be in your thesis.
% To choose one simply delete the others and they will automatically dissappear

% Preface
\preface{Unidentified vessel travelling at sub warp speed, bearing 235.7. Fluctuations in energy readings from it, Captain. All transporters off. A strange set-up, but I'd say the graviton generator is depolarized. The dark colourings of the scrapes are the leavings of natural rubber, a type of non-conductive sole used by researchers experimenting with electricity. The molecules must have been partly de-phased by the anyon beam.
}

\prologue{
Could someone survive inside a transporter buffer for 75 years? Some days you get the bear, and some days the bear gets you. We could cause a diplomatic crisis. Take the ship into the Neutral Zone We finished our first sensor sweep of the neutral zone. Worf, It's better than music. It's jazz. I am your worst nightmare!
}

\foreword[2]{
Sure. You'd be surprised how far a hug goes with Geordi, or Worf. That might've been one of the shortest assignments in the history of Starfleet. Well, I'll say this for him - he's sure of himself.
}
% ----------------------------------------------------------------------

% commands to show or hide front matter pages

\showcopyright
\showabstract
\showcommitteepage
\showdedication
\showacknowledgments
\showpreface
\showprologue
\showforeword
% ------------ TABLE OF CONTENTS ----------------------
% Commands to hide or show lists of figures, tables, etc.
\showlistoffigures
\showlistoftables
\hidenomenclature

% --------- ACRONYMS and SYMBOLS ------------------------------

% Definition of any abbreviations used.
\abbreviations{
    \acro{CRTBP}{Circular Restricted Three Body Problem}
    \acro{NSA}{National Security Agency}
    \acro{SSME}{Space Shuttle Main Engine}
}
% call an abbreviation using \ac{abbrev}

% symbols and acronyms only show up when used in the text
\symbols{
    \acro{F}{Force}    
    \acro{J}{Moment of Inertia}
}       

\hidelistofabbreviations
\hidelistofsymbols

% GLOSSARIES package options - automatically turns off front pages from acronym package

% acronymns and symbols are basically the same, but there are two provided 
% locations where they can show up
\setabbreviationstyle{long-short}
\makeglossaries

\showglossarieslistofabbreviations
\showglossarieslistofsymbols

\newacronym{crtbp}{CRTBP}{Circular Restricted Three Body Problem}
\newacronym{force}{\ensuremath{F}}{force}

\newglossaryentry{F}{
    type=symbols,
    name={\ensuremath{F}},
    sort=F,
    description={External Force}}

    % use either \acrlong (\acl), \acrfull (\acf}, \acrshort (\acs)
% Some abstract text
\abstract{
In mattis lacinia semper. 
Integer eu purus non felis varius dictum tristique sed leo. 
Integer gravida rutrum quam. 
Etiam non posuere nisl. 
Phasellus laoreet sem eget dui commodo pharetra. 
Maecenas eget enim tellus. 
Vivamus rutrum tortor nulla, nec efficitur orci faucibus nec. 
Nunc nec tempus nulla. 
Ut dictum, tellus sed fermentum pharetra, arcu nulla pharetra lectus, sit amet volutpat lorem tortor vitae nulla. 
Interdum et malesuada fames ac ante ipsum primis in faucibus. 
Vivamus rhoncus fermentum turpis et lacinia. 
Vestibulum condimentum molestie odio quis blandit. 
Curabitur a tellus eu ante rutrum finibus eget id libero. 
Suspendisse euismod pretium pretium. 
Maecenas congue interdum ante, ut condimentum velit suscipit at. 
Vestibulum lobortis et orci non maximus.
}
