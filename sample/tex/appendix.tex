% !TEX root = ../thesis-sample.tex
\appendix
\chapter{Methods}
Here is how to implement the methods.

\section{Bisection}
The easiest method.

\begin{equation}\label{eq:sum}
    x_k = \frac{a_k+b_k}{2}
\end{equation}

\section{False Position}
The next one.


\chapter{Using Appendices}    \label{app:appendix}
This appendix contains the portion of the users' manual that describes
how to use appendices with this template.  It is put in this appendix
rather than in Chapter \ref{chap:manual} simply so that there are two
appendices, so that a list of appendices can appear earlier in the
document.

\section{Starting the Appendices}
Actually, using appendices is quite simple.  Immediately after the end
of the last chapter and before the start of the first appendix, simply
enter the command \verb|\appendix|.  This will tell \LaTeX~to change
how it interprets the commands \verb|\chapter|, \verb|\section|,
\textit{etc.}

Each appendix is actually a chapter, so once the \verb|\appendix|
command has been called, start a new appendix by simply using the
\verb|\chapter| command.

Note that the \verb|\appendix| command should be called only
once--not before the start of each appendix.

\section{Lists Including the Appendices}
As mentioned in Section \ref{ssec:lists}, the command

must appear in the preamble if there are more than one appendices.  For
some reason, Rackham does not want the individual appendices and their
sections to appear in the Table of Contents, so a special List of
Appendices page (which must occur in the Table of Contents!) is required
as a sort of extension to the Table of Contents.

This does not require a user of this template to do anything, but it is
so silly that I felt it was worth explaining.  Also, there is nowhere
for the sections or subsections of appendices to show up in the Table of
Contents or any of the lists, but they do still create navigation tabs
in a modern PDF viewer.
